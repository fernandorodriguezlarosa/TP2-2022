\documentclass[]{elsarticle} %review=doublespace preprint=single 5p=2 column
%%% Begin My package additions %%%%%%%%%%%%%%%%%%%

\usepackage[hyphens]{url}


\usepackage{lineno} % add

\usepackage{graphicx}
%%%%%%%%%%%%%%%% end my additions to header

\usepackage[T1]{fontenc}
\usepackage{lmodern}
\usepackage{amssymb,amsmath}
\usepackage{ifxetex,ifluatex}
\usepackage{fixltx2e} % provides \textsubscript
% use upquote if available, for straight quotes in verbatim environments
\IfFileExists{upquote.sty}{\usepackage{upquote}}{}
\ifnum 0\ifxetex 1\fi\ifluatex 1\fi=0 % if pdftex
  \usepackage[utf8]{inputenc}
\else % if luatex or xelatex
  \usepackage{fontspec}
  \ifxetex
    \usepackage{xltxtra,xunicode}
  \fi
  \defaultfontfeatures{Mapping=tex-text,Scale=MatchLowercase}
  \newcommand{\euro}{€}
\fi
% use microtype if available
\IfFileExists{microtype.sty}{\usepackage{microtype}}{}
\usepackage[]{natbib}
\bibliographystyle{plainnat}

\usepackage{graphicx}
\ifxetex
  \usepackage[setpagesize=false, % page size defined by xetex
              unicode=false, % unicode breaks when used with xetex
              xetex]{hyperref}
\else
  \usepackage[unicode=true]{hyperref}
\fi
\hypersetup{breaklinks=true,
            bookmarks=true,
            pdfauthor={},
            pdftitle={Trabajo En Clase TyHM I},
            colorlinks=false,
            urlcolor=blue,
            linkcolor=magenta,
            pdfborder={0 0 0}}

\setcounter{secnumdepth}{0}
% Pandoc toggle for numbering sections (defaults to be off)
\setcounter{secnumdepth}{0}

% Pandoc syntax highlighting
\usepackage{color}
\usepackage{fancyvrb}
\newcommand{\VerbBar}{|}
\newcommand{\VERB}{\Verb[commandchars=\\\{\}]}
\DefineVerbatimEnvironment{Highlighting}{Verbatim}{commandchars=\\\{\}}
% Add ',fontsize=\small' for more characters per line
\usepackage{framed}
\definecolor{shadecolor}{RGB}{248,248,248}
\newenvironment{Shaded}{\begin{snugshade}}{\end{snugshade}}
\newcommand{\AlertTok}[1]{\textcolor[rgb]{0.94,0.16,0.16}{#1}}
\newcommand{\AnnotationTok}[1]{\textcolor[rgb]{0.56,0.35,0.01}{\textbf{\textit{#1}}}}
\newcommand{\AttributeTok}[1]{\textcolor[rgb]{0.77,0.63,0.00}{#1}}
\newcommand{\BaseNTok}[1]{\textcolor[rgb]{0.00,0.00,0.81}{#1}}
\newcommand{\BuiltInTok}[1]{#1}
\newcommand{\CharTok}[1]{\textcolor[rgb]{0.31,0.60,0.02}{#1}}
\newcommand{\CommentTok}[1]{\textcolor[rgb]{0.56,0.35,0.01}{\textit{#1}}}
\newcommand{\CommentVarTok}[1]{\textcolor[rgb]{0.56,0.35,0.01}{\textbf{\textit{#1}}}}
\newcommand{\ConstantTok}[1]{\textcolor[rgb]{0.00,0.00,0.00}{#1}}
\newcommand{\ControlFlowTok}[1]{\textcolor[rgb]{0.13,0.29,0.53}{\textbf{#1}}}
\newcommand{\DataTypeTok}[1]{\textcolor[rgb]{0.13,0.29,0.53}{#1}}
\newcommand{\DecValTok}[1]{\textcolor[rgb]{0.00,0.00,0.81}{#1}}
\newcommand{\DocumentationTok}[1]{\textcolor[rgb]{0.56,0.35,0.01}{\textbf{\textit{#1}}}}
\newcommand{\ErrorTok}[1]{\textcolor[rgb]{0.64,0.00,0.00}{\textbf{#1}}}
\newcommand{\ExtensionTok}[1]{#1}
\newcommand{\FloatTok}[1]{\textcolor[rgb]{0.00,0.00,0.81}{#1}}
\newcommand{\FunctionTok}[1]{\textcolor[rgb]{0.00,0.00,0.00}{#1}}
\newcommand{\ImportTok}[1]{#1}
\newcommand{\InformationTok}[1]{\textcolor[rgb]{0.56,0.35,0.01}{\textbf{\textit{#1}}}}
\newcommand{\KeywordTok}[1]{\textcolor[rgb]{0.13,0.29,0.53}{\textbf{#1}}}
\newcommand{\NormalTok}[1]{#1}
\newcommand{\OperatorTok}[1]{\textcolor[rgb]{0.81,0.36,0.00}{\textbf{#1}}}
\newcommand{\OtherTok}[1]{\textcolor[rgb]{0.56,0.35,0.01}{#1}}
\newcommand{\PreprocessorTok}[1]{\textcolor[rgb]{0.56,0.35,0.01}{\textit{#1}}}
\newcommand{\RegionMarkerTok}[1]{#1}
\newcommand{\SpecialCharTok}[1]{\textcolor[rgb]{0.00,0.00,0.00}{#1}}
\newcommand{\SpecialStringTok}[1]{\textcolor[rgb]{0.31,0.60,0.02}{#1}}
\newcommand{\StringTok}[1]{\textcolor[rgb]{0.31,0.60,0.02}{#1}}
\newcommand{\VariableTok}[1]{\textcolor[rgb]{0.00,0.00,0.00}{#1}}
\newcommand{\VerbatimStringTok}[1]{\textcolor[rgb]{0.31,0.60,0.02}{#1}}
\newcommand{\WarningTok}[1]{\textcolor[rgb]{0.56,0.35,0.01}{\textbf{\textit{#1}}}}

% tightlist command for lists without linebreak
\providecommand{\tightlist}{%
  \setlength{\itemsep}{0pt}\setlength{\parskip}{0pt}}






\begin{document}


\begin{frontmatter}

  \title{Trabajo En Clase TyHM I}
    \author[Facultad de Ingeniería]{Franco Santilli%
  \corref{cor1}%
  \fnref{1}}
   \ead{francosantilli47@gmail.com} 
    \author[Facultad de Ingeniería]{Guillermo Zaina}
   \ead{zainaguillermo@gmail.com} 
    \author[Facultad de Ingeniería]{Marcos Santiago%
  %
  \fnref{2}}
   \ead{marcossantiagoh2@gmail.com} 
    \author[Facultad de Ingeniería]{Renzo Tagarot%
  %
  \fnref{2}}
   \ead{renzot0800@gmail.com} 
    \author[Facultad de Ingeniería]{Fernando Rodriguez%
  %
  \fnref{2}}
   \ead{ferodriguez15@gmail.com} 
      \affiliation[Facultad de Ingeniería]{Paseo Dr.~Emilio Descotte
750, M5500 Mendoza}
    \cortext[cor1]{Corresponding author}
  
  \begin{abstract}
  
  \end{abstract}
  
 \end{frontmatter}

\hypertarget{introducciuxf3n}{%
\subsection{Introducción}\label{introducciuxf3n}}

En este archivo se muestra lo trabajado en clase como modo de
introducción a los conocimientos de R que deberemos aprender para poder
realizar la entrega final, fragmentos de códigos y que hacen y como
funcionan

\hypertarget{fragmentos-de-cuxf3digos}{%
\subsection{Fragmentos de Códigos}\label{fragmentos-de-cuxf3digos}}

\begin{Shaded}
\begin{Highlighting}[]
\NormalTok{A}\OtherTok{\textless{}{-}}\DecValTok{0}
\NormalTok{B}\OtherTok{\textless{}{-}}\DecValTok{1}
\NormalTok{F[}\DecValTok{1}\NormalTok{]}\OtherTok{\textless{}{-}}\NormalTok{A}
\NormalTok{F[}\DecValTok{2}\NormalTok{]}\OtherTok{\textless{}{-}}\NormalTok{B}
\ControlFlowTok{for}\NormalTok{ (i }\ControlFlowTok{in} \DecValTok{3}\SpecialCharTok{:}\DecValTok{100}\NormalTok{) \{ F[i] }\OtherTok{\textless{}{-}}\NormalTok{ (F[i}\DecValTok{{-}1}\NormalTok{]}\SpecialCharTok{+}\NormalTok{F[i}\DecValTok{{-}2}\NormalTok{]) \}}
\FunctionTok{head}\NormalTok{ (F)}
\end{Highlighting}
\end{Shaded}

\begin{verbatim}
## [1] 0 1 1 2 3 5
\end{verbatim}

\begin{Shaded}
\begin{Highlighting}[]
\NormalTok{x}\OtherTok{\textless{}{-}}\FunctionTok{rnorm}\NormalTok{(}\DecValTok{100}\NormalTok{,}\DecValTok{50}\NormalTok{,}\DecValTok{25}\NormalTok{)}
\NormalTok{burbuja }\OtherTok{\textless{}{-}} \ControlFlowTok{function}\NormalTok{(x)\{}
\NormalTok{  n}\OtherTok{\textless{}{-}}\FunctionTok{length}\NormalTok{(x)}
  \ControlFlowTok{for}\NormalTok{ (j }\ControlFlowTok{in} \DecValTok{1}\SpecialCharTok{:}\NormalTok{(n}\DecValTok{{-}1}\NormalTok{)) \{}
    \ControlFlowTok{for}\NormalTok{ (i }\ControlFlowTok{in} \DecValTok{1}\SpecialCharTok{:}\NormalTok{(n}\SpecialCharTok{{-}}\NormalTok{j)) \{}
      \ControlFlowTok{if}\NormalTok{ (x[i]}\SpecialCharTok{\textgreater{}}\NormalTok{x[i}\SpecialCharTok{+}\DecValTok{1}\NormalTok{]) \{}
\NormalTok{        temp}\OtherTok{\textless{}{-}}\NormalTok{x[i]}
\NormalTok{        x[i]}\OtherTok{\textless{}{-}}\NormalTok{x[i}\SpecialCharTok{+}\DecValTok{1}\NormalTok{]}
\NormalTok{        x[i}\SpecialCharTok{+}\DecValTok{1}\NormalTok{]}\OtherTok{\textless{}{-}}\NormalTok{temp}
\NormalTok{      \}}
\NormalTok{    \}}
\NormalTok{  \}}
  \FunctionTok{return}\NormalTok{(x)}
\NormalTok{\}}
\NormalTok{res}\OtherTok{\textless{}{-}}\FunctionTok{burbuja}\NormalTok{(x)}
\NormalTok{res}
\end{Highlighting}
\end{Shaded}

\begin{verbatim}
##   [1]  -0.9955975   3.7762888   6.7675013  11.1163895  12.3716844  13.1919961
##   [7]  13.3235577  13.9882835  14.9464894  17.0024155  17.2823846  19.8319228
##  [13]  20.7407452  26.5450244  26.6651673  26.8584679  26.9035641  29.2543837
##  [19]  29.5005094  30.3653108  30.5200233  31.6349098  32.6137874  33.5095589
##  [25]  33.5540416  33.7823651  34.0849818  34.7108244  34.8191494  36.5633122
##  [31]  37.3269203  37.7295340  37.8636457  38.8057012  40.6706716  43.3481358
##  [37]  43.4034925  43.8193740  43.8973269  45.5642987  45.8492078  47.9877809
##  [43]  48.0796002  48.9344218  49.3184416  49.8699272  51.1979948  51.4383106
##  [49]  51.6791279  51.9140593  52.2237647  53.0105818  53.0483727  53.2638780
##  [55]  54.0587214  54.2859192  55.1423438  55.8238406  56.4796571  58.2861261
##  [61]  58.6620270  58.9426201  59.0187627  59.7637534  60.7078165  61.0700320
##  [67]  61.7135935  62.8429723  63.6349436  64.9877460  68.8069497  69.2966189
##  [73]  71.0958972  72.0087430  72.3313963  72.7570827  73.2754944  73.3878099
##  [79]  74.2040067  74.9742332  75.2200967  75.4997884  75.8548063  75.9915039
##  [85]  76.4902495  77.3575143  78.7751203  79.0690415  79.1522729  82.3537593
##  [91]  83.2717399  86.6115419  88.3162479  91.4907457  94.1650412  97.1811589
##  [97]  98.0851080 102.6618328 107.4188643 114.9425602
\end{verbatim}

\begin{Shaded}
\begin{Highlighting}[]
\NormalTok{t0}\OtherTok{\textless{}{-}}\FunctionTok{Sys.time}\NormalTok{()}
\NormalTok{x}\OtherTok{\textless{}{-}}\FunctionTok{rnorm}\NormalTok{(}\DecValTok{100}\NormalTok{,}\DecValTok{50}\NormalTok{,}\DecValTok{25}\NormalTok{)}
\NormalTok{burbuja }\OtherTok{\textless{}{-}} \ControlFlowTok{function}\NormalTok{(x)\{}
\NormalTok{  n}\OtherTok{\textless{}{-}}\FunctionTok{length}\NormalTok{(x)}
  \ControlFlowTok{for}\NormalTok{ (j }\ControlFlowTok{in} \DecValTok{1}\SpecialCharTok{:}\NormalTok{(n}\DecValTok{{-}1}\NormalTok{)) \{}
    \ControlFlowTok{for}\NormalTok{ (i }\ControlFlowTok{in} \DecValTok{1}\SpecialCharTok{:}\NormalTok{(n}\SpecialCharTok{{-}}\NormalTok{j)) \{}
      \ControlFlowTok{if}\NormalTok{ (x[i]}\SpecialCharTok{\textgreater{}}\NormalTok{x[i}\SpecialCharTok{+}\DecValTok{1}\NormalTok{]) \{}
\NormalTok{        temp}\OtherTok{\textless{}{-}}\NormalTok{x[i]}
\NormalTok{        x[i]}\OtherTok{\textless{}{-}}\NormalTok{x[i}\SpecialCharTok{+}\DecValTok{1}\NormalTok{]}
\NormalTok{        x[i}\SpecialCharTok{+}\DecValTok{1}\NormalTok{]}\OtherTok{\textless{}{-}}\NormalTok{temp}
\NormalTok{      \}}
\NormalTok{    \}}
\NormalTok{  \}}
  \FunctionTok{return}\NormalTok{(x)}
\NormalTok{\}}
\NormalTok{res}\OtherTok{\textless{}{-}}\FunctionTok{burbuja}\NormalTok{(x)}
\NormalTok{res}
\end{Highlighting}
\end{Shaded}

\begin{verbatim}
##   [1]  -2.526493   2.220028   7.546517  14.127133  17.284483  20.599031
##   [7]  22.521082  23.081176  23.261928  24.320832  25.041102  26.293095
##  [13]  31.044062  31.574734  31.944678  32.394472  32.603816  34.013013
##  [19]  34.617305  34.677244  35.338116  35.353994  36.692042  37.088201
##  [25]  37.346985  38.717020  39.848896  39.935346  40.290678  42.143486
##  [31]  42.654941  42.703559  43.582756  43.587337  44.069049  45.075642
##  [37]  45.113091  45.377827  45.694195  45.804847  46.039453  47.849003
##  [43]  48.468012  50.117626  50.177270  50.732179  51.184511  51.250953
##  [49]  51.486522  52.434571  53.398487  54.064371  55.646572  55.765504
##  [55]  55.841154  56.250822  57.713666  58.205436  58.206399  58.583686
##  [61]  59.406904  59.491970  61.205835  61.556379  61.856931  61.996656
##  [67]  63.033385  63.687266  65.155896  66.645569  69.312501  70.069183
##  [73]  70.617032  70.683327  70.845981  71.849792  72.043971  72.704888
##  [79]  74.342622  75.344856  75.651586  76.052872  76.395079  77.223190
##  [85]  78.177295  79.139505  80.280839  82.949626  83.253269  85.567116
##  [91]  85.721124  87.087369  87.498939  87.700550  88.584241  90.569905
##  [97]  92.306352  95.807388  99.993211 102.098806
\end{verbatim}

\begin{Shaded}
\begin{Highlighting}[]
\NormalTok{tf}\OtherTok{\textless{}{-}}\FunctionTok{Sys.time}\NormalTok{()}
\CommentTok{\#ahora medimos la velocidad del algoritmo}
\NormalTok{tf}\SpecialCharTok{{-}}\NormalTok{t0}
\end{Highlighting}
\end{Shaded}

\begin{verbatim}
## Time difference of 0.1551981 secs
\end{verbatim}

\begin{Shaded}
\begin{Highlighting}[]
\FunctionTok{library}\NormalTok{(tictoc)}
\FunctionTok{tic}\NormalTok{()}
\NormalTok{x}\OtherTok{\textless{}{-}}\FunctionTok{rnorm}\NormalTok{(}\DecValTok{100}\NormalTok{,}\DecValTok{50}\NormalTok{,}\DecValTok{25}\NormalTok{)}
\NormalTok{burbuja }\OtherTok{\textless{}{-}} \ControlFlowTok{function}\NormalTok{(x)\{}
\NormalTok{  n}\OtherTok{\textless{}{-}}\FunctionTok{length}\NormalTok{(x)}
  \ControlFlowTok{for}\NormalTok{ (j }\ControlFlowTok{in} \DecValTok{1}\SpecialCharTok{:}\NormalTok{(n}\DecValTok{{-}1}\NormalTok{)) \{}
    \ControlFlowTok{for}\NormalTok{ (i }\ControlFlowTok{in} \DecValTok{1}\SpecialCharTok{:}\NormalTok{(n}\SpecialCharTok{{-}}\NormalTok{j)) \{}
      \ControlFlowTok{if}\NormalTok{ (x[i]}\SpecialCharTok{\textgreater{}}\NormalTok{x[i}\SpecialCharTok{+}\DecValTok{1}\NormalTok{]) \{}
\NormalTok{        temp}\OtherTok{\textless{}{-}}\NormalTok{x[i]}
\NormalTok{        x[i]}\OtherTok{\textless{}{-}}\NormalTok{x[i}\SpecialCharTok{+}\DecValTok{1}\NormalTok{]}
\NormalTok{        x[i}\SpecialCharTok{+}\DecValTok{1}\NormalTok{]}\OtherTok{\textless{}{-}}\NormalTok{temp}
\NormalTok{      \}}
\NormalTok{    \}}
\NormalTok{  \}}
  \FunctionTok{return}\NormalTok{(x)}
\NormalTok{\}}
\NormalTok{res}\OtherTok{\textless{}{-}}\FunctionTok{burbuja}\NormalTok{(x)}
\NormalTok{res}
\end{Highlighting}
\end{Shaded}

\begin{verbatim}
##   [1]  -7.435248  -2.481888  -1.218721   1.632252   6.426259   7.104327
##   [7]   8.609149   9.144819  13.055219  14.771200  15.494357  16.014596
##  [13]  16.266847  17.379334  17.862300  18.773446  19.075061  19.380523
##  [19]  19.620055  24.498302  25.565413  26.832655  27.026582  29.225409
##  [25]  30.703107  30.751534  32.587487  33.663776  33.759393  36.127879
##  [31]  37.708992  37.920027  38.656652  39.484726  40.036221  40.581770
##  [37]  41.000604  41.466927  42.050987  43.281008  43.708797  43.879256
##  [43]  43.937458  43.967858  44.168787  44.816738  44.888452  44.979318
##  [49]  46.723360  47.161659  47.273288  47.578258  48.893119  49.290713
##  [55]  49.305098  50.387461  51.104896  51.863854  51.990642  51.994089
##  [61]  53.180367  54.151726  54.971656  56.214497  57.780036  58.808878
##  [67]  59.427194  59.982873  61.974604  63.256512  65.687290  67.701015
##  [73]  67.815912  67.919376  69.014277  69.837554  70.495729  70.649720
##  [79]  73.487408  73.621006  73.753217  76.880168  77.038830  81.483575
##  [85]  82.917764  85.277221  86.488691  86.765721  87.485280  89.193139
##  [91]  89.687764  90.479060  91.090927  94.670840  95.881095  96.763501
##  [97] 106.443753 111.534408 112.251570 115.203556
\end{verbatim}

\begin{Shaded}
\begin{Highlighting}[]
\FunctionTok{toc}\NormalTok{()}
\end{Highlighting}
\end{Shaded}

\begin{verbatim}
## 0.05 sec elapsed
\end{verbatim}

\begin{Shaded}
\begin{Highlighting}[]
\FunctionTok{library}\NormalTok{(microbenchmark)}
\NormalTok{x}\OtherTok{\textless{}{-}}\FunctionTok{rnorm}\NormalTok{(}\DecValTok{100}\NormalTok{,}\DecValTok{50}\NormalTok{,}\DecValTok{25}\NormalTok{)}
\NormalTok{mbm}\OtherTok{\textless{}{-}}\FunctionTok{microbenchmark}\NormalTok{(}
  \StringTok{"burbuja"}\OtherTok{=}\NormalTok{\{}
\NormalTok{    x}\OtherTok{\textless{}{-}}\FunctionTok{rnorm}\NormalTok{(}\DecValTok{100}\NormalTok{,}\DecValTok{50}\NormalTok{,}\DecValTok{25}\NormalTok{)}
\NormalTok{burbuja }\OtherTok{\textless{}{-}} \ControlFlowTok{function}\NormalTok{(x)\{}
\NormalTok{  n}\OtherTok{\textless{}{-}}\FunctionTok{length}\NormalTok{(x)}
  \ControlFlowTok{for}\NormalTok{ (j }\ControlFlowTok{in} \DecValTok{1}\SpecialCharTok{:}\NormalTok{(n}\DecValTok{{-}1}\NormalTok{)) \{}
    \ControlFlowTok{for}\NormalTok{ (i }\ControlFlowTok{in} \DecValTok{1}\SpecialCharTok{:}\NormalTok{(n}\SpecialCharTok{{-}}\NormalTok{j)) \{}
      \ControlFlowTok{if}\NormalTok{ (x[i]}\SpecialCharTok{\textgreater{}}\NormalTok{x[i}\SpecialCharTok{+}\DecValTok{1}\NormalTok{]) \{}
\NormalTok{        temp}\OtherTok{\textless{}{-}}\NormalTok{x[i]}
\NormalTok{        x[i]}\OtherTok{\textless{}{-}}\NormalTok{x[i}\SpecialCharTok{+}\DecValTok{1}\NormalTok{]}
\NormalTok{        x[i}\SpecialCharTok{+}\DecValTok{1}\NormalTok{]}\OtherTok{\textless{}{-}}\NormalTok{temp}
\NormalTok{      \}}
\NormalTok{    \}}
\NormalTok{  \}}
  \FunctionTok{return}\NormalTok{(x)}
\NormalTok{\}}
\NormalTok{res}\OtherTok{\textless{}{-}}\FunctionTok{burbuja}\NormalTok{(x)}
\NormalTok{res}
\NormalTok{  \},}
\StringTok{"sort"}\OtherTok{=}\NormalTok{\{}
  \FunctionTok{sort}\NormalTok{(x)}
\NormalTok{\}}
\NormalTok{)}

\NormalTok{mbm}
\end{Highlighting}
\end{Shaded}

\begin{verbatim}
## Unit: microseconds
##     expr    min      lq     mean  median      uq     max neval
##  burbuja 1246.4 1524.25 1950.711 1623.45 1812.45 22661.8   100
##     sort   88.9  101.35  211.617  118.75  280.60  1279.3   100
\end{verbatim}

\begin{Shaded}
\begin{Highlighting}[]
\FunctionTok{library}\NormalTok{(ggplot2)}
\FunctionTok{autoplot}\NormalTok{(mbm)}
\end{Highlighting}
\end{Shaded}

\begin{verbatim}
## Coordinate system already present. Adding new coordinate system, which will replace the existing one.
\end{verbatim}

\includegraphics{Trabajo_En_Clase__files/figure-latex/unnamed-chunk-6-1.pdf}

\hypertarget{creaciuxf3n-de-vectores}{%
\subsection{creación de vectores}\label{creaciuxf3n-de-vectores}}

\begin{Shaded}
\begin{Highlighting}[]
\NormalTok{v1 }\OtherTok{\textless{}{-}} \FunctionTok{c}\NormalTok{(}\DecValTok{1}\NormalTok{,}\DecValTok{2}\NormalTok{,}\DecValTok{3}\NormalTok{,}\DecValTok{4}\NormalTok{,}\DecValTok{5}\NormalTok{)}
\end{Highlighting}
\end{Shaded}

creación de un vector de 9 componentes

\begin{Shaded}
\begin{Highlighting}[]
\NormalTok{v2 }\OtherTok{\textless{}{-}} \FunctionTok{c}\NormalTok{(}\DecValTok{1}\NormalTok{,}\DecValTok{2}\NormalTok{,}\DecValTok{3}\NormalTok{,}\DecValTok{4}\NormalTok{,}\DecValTok{5}\NormalTok{,}\DecValTok{6}\NormalTok{,}\DecValTok{7}\NormalTok{,}\DecValTok{8}\NormalTok{,}\DecValTok{9}\NormalTok{)}
\end{Highlighting}
\end{Shaded}

\hypertarget{creaciuxf3n-de-matrices}{%
\subsection{creación de matrices}\label{creaciuxf3n-de-matrices}}

\begin{Shaded}
\begin{Highlighting}[]
\NormalTok{m1}\OtherTok{\textless{}{-}} \FunctionTok{matrix}\NormalTok{(v2,}\AttributeTok{ncol=}\DecValTok{3}\NormalTok{,}\AttributeTok{byrow=}\ConstantTok{FALSE}\NormalTok{)}
\end{Highlighting}
\end{Shaded}

el byrow me deja ordenar los valores por fila en el caso de TRUE, o en
columna en el caso de FALSE

\hypertarget{averiguar-que-clase-de-ojeto-hemos-creado}{%
\subsection{averiguar que clase de ojeto hemos
creado}\label{averiguar-que-clase-de-ojeto-hemos-creado}}

para saber de qué clase se utiliza el comando class(nombre del objeto)
vemos que nos dice que el vector es de tipo numérico y la matriz de tipo
array o matriz, además siempre es numerica la matriz.

\begin{Shaded}
\begin{Highlighting}[]
\FunctionTok{class}\NormalTok{(v1)}
\end{Highlighting}
\end{Shaded}

\begin{verbatim}
## [1] "numeric"
\end{verbatim}

\begin{Shaded}
\begin{Highlighting}[]
\FunctionTok{class}\NormalTok{(m1)}
\end{Highlighting}
\end{Shaded}

\begin{verbatim}
## [1] "matrix" "array"
\end{verbatim}

\hypertarget{creaciuxf3n-de-un-vector-de-palabras}{%
\subsection{creación de un vector de
palabras}\label{creaciuxf3n-de-un-vector-de-palabras}}

\begin{Shaded}
\begin{Highlighting}[]
\NormalTok{v3}\OtherTok{\textless{}{-}} \FunctionTok{c}\NormalTok{(}\StringTok{"a"}\NormalTok{,}\StringTok{"b"}\NormalTok{,}\StringTok{"c"}\NormalTok{)}
\FunctionTok{class}\NormalTok{(v3)}
\end{Highlighting}
\end{Shaded}

\begin{verbatim}
## [1] "character"
\end{verbatim}

\begin{Shaded}
\begin{Highlighting}[]
\NormalTok{v3}
\end{Highlighting}
\end{Shaded}

\begin{verbatim}
## [1] "a" "b" "c"
\end{verbatim}

hay un comando que se llama dimnames que sirve para ponerle nombre a las
filas y las columnas.

\hypertarget{importar-datos-de-la-red-o-de-excel}{%
\subsection{importar datos de la red o de
excel}\label{importar-datos-de-la-red-o-de-excel}}

lo que nos dice al pegar un dato de excel es que enumera todos los
cambios que tuvimos que hacer en el archivo original para dejarlo
ordenado y acomodado.

\begin{Shaded}
\begin{Highlighting}[]
\FunctionTok{library}\NormalTok{(readr)}
\NormalTok{casos }\OtherTok{\textless{}{-}} \FunctionTok{read\_delim}\NormalTok{(}\StringTok{"C:/Users/franc/Downloads/casos.csv"}\NormalTok{, }
    \AttributeTok{delim =} \StringTok{";"}\NormalTok{, }\AttributeTok{escape\_double =} \ConstantTok{FALSE}\NormalTok{, }\AttributeTok{col\_types =} \FunctionTok{cols}\NormalTok{(}\StringTok{\textasciigrave{}}\AttributeTok{Covid Argentina}\StringTok{\textasciigrave{}} \OtherTok{=} \FunctionTok{col\_date}\NormalTok{(}\AttributeTok{format =} \StringTok{"\%m/\%d/\%Y"}\NormalTok{)), }
    \AttributeTok{trim\_ws =} \ConstantTok{TRUE}\NormalTok{)}
\end{Highlighting}
\end{Shaded}

\begin{verbatim}
## New names:
## * `` -> `...2`
## * `` -> `...3`
\end{verbatim}

\begin{verbatim}
## Warning: One or more parsing issues, see `problems()` for details
\end{verbatim}

dataset es un conjunto de datos de una tabla tomados de la vida real que
estan documentados y estan guardados en repositorios de datos.

\hypertarget{ploteo-de-datos}{%
\subsection{ploteo de datos}\label{ploteo-de-datos}}

\begin{Shaded}
\begin{Highlighting}[]
\NormalTok{casos}\SpecialCharTok{$}\NormalTok{...}\DecValTok{2}
\end{Highlighting}
\end{Shaded}

\begin{verbatim}
##  [1] "Casos" "1"     "2"     "2"     "12"    "17"    "19"    "21"    "31"   
## [10] "34"    "45"    "56"    "65"    "79"    "98"    "128"   "158"   "225"  
## [19] "266"   "301"   "387"   "502"   "589"   "690"   "745"   "820"   "1054" 
## [28] "1054"  "1133"  "1265"  "1353"  "1451"  "1554"  "1628"  "1715"
\end{verbatim}

\begin{Shaded}
\begin{Highlighting}[]
\FunctionTok{plot}\NormalTok{(casos}\SpecialCharTok{$}\NormalTok{...}\DecValTok{2}\NormalTok{,}\AttributeTok{main=}\StringTok{"Contagios 2020"}\NormalTok{,}\AttributeTok{ylab=}\StringTok{"Semana"}\NormalTok{,}\AttributeTok{xlab=}\StringTok{"Casos Positivos"}\NormalTok{)}
\end{Highlighting}
\end{Shaded}

\begin{verbatim}
## Warning in xy.coords(x, y, xlabel, ylabel, log): NAs introducidos por coerción
\end{verbatim}

\includegraphics{Trabajo_En_Clase__files/figure-latex/unnamed-chunk-13-1.pdf}


\end{document}
